\FILENAME\

\section{reStructuredText}\label{restructuredtext}

reStructuredText (RST) purpose is to provide an easy-to-read,
what-you-see-is-what-you-get plaintext markup syntax and parser system.
With its help you can develop documentation not only for stand aone
documentation, simple web pages, an in-line program documentation (such
as Python). RST is extensible and new features can be added. It is used
in sphinx as one of its supported formats.

\subsection{Links}\label{links}

\begin{itemize}

\item
  RST Sphinx documentation:
  \url{http://www.sphinx-doc.org/en/stable/rest.html}
\item
  RST Syntax: \url{http://docutils.sourceforge.net/rst.html}
\item
  Important extensions: \url{http://sphinx-doc.org/ext/todo.html}
\end{itemize}

Cheatcheat:
\smallskip

  \URL{http://github.com/ralsina/rst-cheatsheet/raw/master/rst-cheatsheet.pdf}
  \URL{http://docutils.sourceforge.net/docs/ref/rst/directives.html}

\subsection{Source}\label{source}

The source for this page is located at

  \URL{https://raw.githubusercontent.com/cloudmesh/classes/master/docs/source/lesson/doc/rst.rst}

This way you can look at the source on how we create this page.

\subsection{Sections}\label{sections}

\# with overline, for parts \verb|*| with overline, for chapters \verb|=|, for
sections \verb|-|, for subsections \verb|^|, for subsubsections \verb|"|, for paragraphs

RST allows to specify a number of sections. You can do this with the
various underlines:

\begin{verbatim}
*********************
Chapter
*********************
Section
=====================
Subsection
---------------------
Subsubsection
^^^^^^^^^^^^^^^^^^^^^
Paragraph
~~~~~~~~~~~~~~~~~~~~~
\end{verbatim}

\subsection{Listtable}\label{listtable}

\begin{verbatim}
.. csv-table:: Eye colors
   :header: "Name", "Firstname", "eyes"
   :widths: 20, 20, 10

   "von Laszewski", "Gregor", "gray"
\end{verbatim}

\subsection{Exceltable}\label{exceltable}

we have integrated Excel table from
\url{http://pythonhosted.org//sphinxcontrib-exceltable/} intou our
sphinx allowing the definition of more elaborate tables specified in
excel. Howere the most convenient way may be to use list-tables. The
documentation to list tables can be found at
\url{http://docutils.sourceforge.net/docs/ref/rst/directives.html\#list-table}

\subsection{Boxes}\label{boxes}

\subsubsection{Seealso}\label{seealso}

\begin{verbatim}
.. seealso:: This is a simple **seealso** note. 
\end{verbatim}

\subsubsection{Note}\label{note}

This is a \textbf{note} box.

\begin{verbatim}
.. note::  This is a **note** box.
\end{verbatim}

\subsubsection{Warning}\label{warning}

note the space between the directive and the text

\begin{verbatim}
.. warning:: note the space between the directive and the text
\end{verbatim}

\subsubsection{Others}\label{others}

This is an \textbf{attention} box.

\begin{verbatim}
.. attention:: This is an **attention** box.
\end{verbatim}

This is a \textbf{caution} box.

\begin{verbatim}
.. caution:: This is a **caution** box.
\end{verbatim}

This is a \textbf{danger} box.

\begin{verbatim}
.. danger:: This is a **danger** box.
\end{verbatim}

This is a \textbf{error} box.

\begin{verbatim}
.. error:: This is a **error** box.
\end{verbatim}

This is a \textbf{hint} box.

\begin{verbatim}
.. hint:: This is a **hint** box.
\end{verbatim}

This is an \textbf{important} box.

\begin{verbatim}
.. important:: This is an **important** box.
\end{verbatim}

This is a \textbf{tip} box.

\begin{verbatim}
.. tip:: This is a **tip** box.
\end{verbatim}

\subsection{Sidebar directive}\label{sidebar-directive}

It is possible to create sidebar using the following code:

\begin{verbatim}
.. sidebar:: Sidebar Title
    :subtitle: Optional Sidebar Subtitle

    Subsequent indented lines comprise
    the body of the sidebar, and are
    interpreted as body elements.
\end{verbatim}

\textbf{Sidebar Title: Optional Sidebar Subtitle}

Subsequent indented lines comprise the body of the sidebar, and are
interpreted as body elements.

\subsection{Sphinx Prompt}\label{sphinx-prompt}

\begin{verbatim}
.. prompt:: bash, cloudmesh$

   wget -O cm-setup.sh http://bit.ly/cloudmesh-client-xenial
   sh cm-setup.sh
\end{verbatim}

\subsection{Programm examples}\label{programm-examples}

You can include code examples and bash commands with two colons.

This is an example for python:

\begin{verbatim}
print ("Hallo World")
\end{verbatim}

This is an example for a shell command:

\begin{verbatim}
$ ls -lisa
\end{verbatim}

\subsection{Hyperlinks}\label{hyperlinks}

Direct links to html pages can ve done with:

\begin{verbatim}
`This is a link to an html page <hadoop.html>`_
\end{verbatim}

Note that this page could be generated from an rst page

Links to the FG portal need to be formulated with the portal tag:

\begin{verbatim}
:portal:`List to FG projects </projects/all>`
\end{verbatim}

In case a subsection has a link declared you can use \verb|:ref:|. This is the
prefered way as it can be used to point even to subsections:

\begin{verbatim}
:ref:`Connecting private network VMs  clusters <_s_vpn>` 
\end{verbatim}

A html link can be created anywhere in the document but must be unique.
for example if you place:

\begin{verbatim}
.. _s_vpn:
\end{verbatim}

in the text it will create a target to which the above link points when
you click on it

\subsection{Todo}\label{todo}

\begin{verbatim}
.. todo:: an example
\end{verbatim}

\includegraphics[width=\columnwidth]{images/todo.png}
