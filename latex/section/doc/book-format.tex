
\chapterimage{water.png} % Chapter heading image

\chapter{Book Format}

\FILENAME\

This document is prepared using a special enhanced \LaTeX~book format to allow
easy integration for section and index references, as well as
citations and other straight forward enhancements that makes it easy
for students to obtain a document with much of the Lecture material
included. We provide some examples so that if you like to contribute
you can do so easily and can use the correct way of doing so.


\section{Paragraphs of Text}\index{\LaTeX!Book!Paragraphs of Text}

\lipsum[1] % Dummy text

%------------------------------------------------

\section{Citation}\index{\LaTeX!Book!Citation}

\begin{tcblisting}{colback=blue!5!white,colframe=gray!50!blue,listing side text,
  title=Numbered List,fonttitle=\bfseries}
This statement requires citation \cite{article_key}; 
this one is more specific \cite[162]{book_key}.
\end{tcblisting}

%------------------------------------------------

\section{Lists}\index{\LaTeX!Book!Lists}

Lists are useful to present information in a concise and/or ordered way\footnote{Footnote example...}.

\subsection{Numbered List}\index{\LaTeX!Book!Lists!Numbered List}

\begin{tcblisting}{colback=blue!5!white,colframe=gray!50!blue,listing side text,
  title=Numbered List,fonttitle=\bfseries}
\begin{enumerate}
\item The first item
\item The second item
\item The third item
\end{enumerate}
\end{tcblisting}

\subsection{Bullet Points}\index{\LaTeX!Book!Lists!Bullet Points}

\begin{tcblisting}{colback=blue!5!white,colframe=gray!50!blue,listing side text,
  title=Bullet Points,fonttitle=\bfseries}
\begin{itemize}
\item The first item
\item The second item
\item The third item
\end{itemize}
\end{tcblisting}

\subsection{Descriptions and Definitions}\index{\LaTeX!Book!Lists!Descriptions and Definitions}

\begin{tcblisting}{colback=blue!5!white,colframe=gray!50!blue,listing side text,
  title=Enumerations,fonttitle=\bfseries}
\begin{description}
\item[Name] Description
\item[Word] Definition
\item[Comment] Elaboration
\end{description}
\end{tcblisting}

%-------------------------------------------------------------------------------
%	CHAPTER 2
%-------------------------------------------------------------------------------


\section{Theorems}\index{\LaTeX!Book!Theorems}

This is an example of theorems.

\subsection{Several equations}\index{\LaTeX!Book!Theorems!Several Equations}
This is a theorem consisting of several equations.

\begin{lstlisting}
\begin{theorem}[Name of the theorem]
In $E=\mathbb{R}^n$ all norms are equivalent. It has the properties:
\begin{align}
& \big| ||\mathbf{x}|| - ||\mathbf{y}|| \big|\leq || \mathbf{x}- \mathbf{y}||\\
&  ||\sum_{i=1}^n\mathbf{x}_i||\leq \sum_{i=1}^n||\mathbf{x}_i||\quad\text{where $n$ is a finite integer}
\end{align}
\end{theorem}
\end{lstlisting}

\begin{theorem}[Name of the theorem]
In $E=\mathbb{R}^n$ all norms are equivalent. It has the properties:
\begin{align}
& \big| ||\mathbf{x}|| - ||\mathbf{y}|| \big|\leq || \mathbf{x}- \mathbf{y}||\\
&  ||\sum_{i=1}^n\mathbf{x}_i||\leq \sum_{i=1}^n||\mathbf{x}_i||\quad\text{where $n$ is a finite integer}
\end{align}
\end{theorem}

\subsection{Single Line}\index{\LaTeX!Book!Theorems!Single Line}
This is a theorem consisting of just one line.

\begin{verbatim}
\begin{theorem}
A set $\mathcal{D}(G)$ in dense in $L^2(G)$, $|\cdot|_0$. 
\end{theorem}
\end{verbatim}

\begin{theorem}
A set $\mathcal{D}(G)$ in dense in $L^2(G)$, $|\cdot|_0$. 
\end{theorem}

%------------------------------------------------

\section{Definitions}\index{\LaTeX!Book!Definitions}

This is an example of a definition. A definition could be mathematical or it could define a concept.

\begin{definition}[Definition name]
Given a vector space $E$, a norm on $E$ is an application, denoted $||\cdot||$, $E$ in $\mathbb{R}^+=[0,+\infty[$ such that:
\begin{align}
& ||\mathbf{x}||=0\ \Rightarrow\ \mathbf{x}=\mathbf{0}\\
& ||\lambda \mathbf{x}||=|\lambda|\cdot ||\mathbf{x}||\\
& ||\mathbf{x}+\mathbf{y}||\leq ||\mathbf{x}||+||\mathbf{y}||
\end{align}
\end{definition}

%------------------------------------------------

\section{Notations}\index{\LaTeX!Book!Notations}

\begin{tcblisting}{colback=blue!5!white,colframe=gray!50!blue,listing side text,  title=Exercise,fonttitle=\bfseries}
\begin{notation}
Given an open subset $G$ of $\mathbb{R}^n$, the set of functions $\varphi$ are:
\begin{enumerate}
\item Bounded support $G$;
\item Infinitely differentiable;
\end{enumerate}
a vector space is denoted by $\mathcal{D}(G)$. 
\end{notation}
\end{tcblisting}

%------------------------------------------------

\section{Remarks}\index{\LaTeX!Book!Remarks}

This is an example of a remark.

\begin{tcblisting}{colback=blue!5!white,colframe=gray!50!blue,listing side text,  title=Exercise,fonttitle=\bfseries}
\begin{remark}
The concepts presented here are now in conventional employment in mathematics. Vector spaces are taken over the field $\mathbb{K}=\mathbb{R}$, however, established properties are easily extended to $\mathbb{K}=\mathbb{C}$.
\end{remark}
\end{tcblisting}

%------------------------------------------------

\section{Corollaries}\index{\LaTeX!Book!Corollaries}

This is an example of a corollary.
\begin{tcblisting}{colback=blue!5!white,colframe=gray!50!blue,listing
    side text,  title=Exercise,fonttitle=\bfseries}
\begin{corollary}[Corollary name]
The concepts presented here are now in conventional employment in mathematics. Vector spaces are taken over the field $\mathbb{K}=\mathbb{R}$, however, established properties are easily extended to $\mathbb{K}=\mathbb{C}$.
\end{corollary}
\end{tcblisting}
%------------------------------------------------

\section{Propositions}\index{\LaTeX!Book!Propositions}

These are examples of propositions.

\subsection{Several equations}\index{\LaTeX!Book!Propositions!Several Equations}


\begin{proposition}[Proposition name]
It has the properties:
\begin{align}
& \big| ||\mathbf{x}|| - ||\mathbf{y}|| \big|\leq || \mathbf{x}- \mathbf{y}||\\
&  ||\sum_{i=1}^n\mathbf{x}_i||\leq \sum_{i=1}^n||\mathbf{x}_i||\quad\text{where $n$ is a finite integer}
\end{align}
\end{proposition}

\subsection{Single Line}\index{\LaTeX!Book!Propositions!Single Line}

\begin{proposition} 
Let $f,g\in L^2(G)$; if $\forall \varphi\in\mathcal{D}(G)$, $(f,\varphi)_0=(g,\varphi)_0$ then $f = g$. 
\end{proposition}

%------------------------------------------------

\section{Examples}\index{\LaTeX!Book!Examples}

This is an example of examples.

\subsection{Equation and Text}\index{\LaTeX!Book!Examples!Equation and Text}

\begin{example}
Let $G=\{x\in\mathbb{R}^2:|x|<3\}$ and denoted by: $x^0=(1,1)$; consider the function:
\begin{equation}
f(x)=\left\{\begin{aligned} & \mathrm{e}^{|x|} & & \text{si $|x-x^0|\leq 1/2$}\\
& 0 & & \text{si $|x-x^0|> 1/2$}\end{aligned}\right.
\end{equation}
The function $f$ has bounded support, we can take $A=\{x\in\mathbb{R}^2:|x-x^0|\leq 1/2+\epsilon\}$ for all $\epsilon\in\intoo{0}{5/2-\sqrt{2}}$.
\end{example}

\subsection{Paragraph of Text}\index{\LaTeX!Book!Examples!Paragraph of Text}

\begin{tcblisting}{colback=blue!5!white,colframe=gray!50!blue,listing side text,  title=Example,fonttitle=\bfseries}
\begin{example}[Example name]
\lipsum[1]
\end{example}
\end{tcblisting}

%------------------------------------------------

\section{Exercises}\index{\LaTeX!Book!Exercises}

This is an example of an exercise.

\begin{tcblisting}{colback=blue!5!white,colframe=gray!50!blue,listing side text,  title=Exercise,fonttitle=\bfseries}
\begin{exercise}
This is a good place to ask a question to test learning progress or further cement ideas into students' minds.
\end{exercise}
\end{tcblisting}

%------------------------------------------------

\section{Problems}\index{\LaTeX!Book!Problems}

\begin{tcblisting}{colback=blue!5!white,colframe=gray!50!blue,listing side text,  title=Exercise,fonttitle=\bfseries}
\begin{problem}
What is the average airspeed velocity of an unladen swallow?
\end{problem}
\end{tcblisting}

%------------------------------------------------

\section{Vocabulary}\index{\LaTeX!Book!Vocabulary}

Define a word to improve a students' vocabulary.

\begin{tcblisting}{colback=blue!5!white,colframe=gray!50!blue,listing side text,  title=Exercise,fonttitle=\bfseries}
\begin{vocabulary}[Word]
Definition of word.
\end{vocabulary}
\end{tcblisting}

\section{Tables}\index{\LaTeX!Book!Table}

\begin{table}[h]
\centering
\begin{tabular}{l l l}
\toprule
\textbf{Treatments} & \textbf{Response 1} & \textbf{Response 2}\\
\midrule
Treatment 1 & 0.0003262 & 0.562 \\
Treatment 2 & 0.0015681 & 0.910 \\
Treatment 3 & 0.0009271 & 0.296 \\
\bottomrule
\end{tabular}
\caption{Table caption}
\end{table}

%------------------------------------------------

\section{Figures}\index{\LaTeX!Book!Figure}

\begin{figure}[h]
\centering\includegraphics[scale=0.5]{placeholder}
\caption{Figure caption}
\end{figure}

\subsection{Colored Boxes}
\index{\LaTeX!Book!Boxes}
\index{\LaTeX!Book!NOTE}
\index{\LaTeX!Book!WARNING}
\index{\LaTeX!Book!IU}

The package \verb|tcolorbox| provides sophisticated support to include
color boxes. Together with the new environment we can create nice add
ons to for example include notes.

\URL{http://osl.ugr.es/CTAN/macros/latex/contrib/tcolorbox/tcolorbox.pdf}

We have provided in this document notes as follows


\begin{tcblisting}{colback=blue!5!white,colframe=gray!50!blue,listing side text,  title=Note,fonttitle=\bfseries}
\begin{NOTE}
This is a note
\end{NOTE}
\end{tcblisting}

\begin{tcblisting}{colback=blue!5!white,colframe=gray!50!blue,listing side text,  title=Warning,fonttitle=\bfseries}
\begin{WARNING}
This is a note
\end{WARNING}
\end{tcblisting}

\begin{tcblisting}{colback=blue!5!white,colframe=gray!50!blue,listing side text,  title=Warning,fonttitle=\bfseries}
\begin{IU}
This is a note
\end{IU}
\end{tcblisting}

\subsection{Section References}

Section references with the Where command are used to refer to a
section by label, but be presented with the section number, the title,
and the page number. It can be augmented with a text, such as in which
week the section is best to be studied. This is useful for creating a
customized Syllabus

\begin{tcblisting}{colback=blue!5!white,colframe=gray!50!blue,
    listing side text, title=Warning,fonttitle=\bfseries}
\WHERE{\YES}{C:linux}{Week 2}
\end{tcblisting}

\begin{tcblisting}{colback=blue!5!white,colframe=gray!50!blue,
    listing side text, title=Warning,fonttitle=\bfseries}
\WHERE{\NO}{C:linux}{Week 2}
\end{tcblisting}

The checkmark indicate if the lecture has been released for the
ongoing semester.

\subsection{Partial Compile}

As we use include in the different parts, changes will only effect the
appropriate part. However, we can also explicitly set in the preamble
of the document which includes we compile. This is helpful, as the
document is rather large and we may want to focus on just one or more
parts without creating a large document that includes all the parts.
For this we can use the convenient include only command and add to it
with comma separated the parts that we want to compile and are
regularly included within the main document. For example to just work
on the container part, we simply add the following to the preamble.

\begin{verbatim}
\documentclass{format/laszewski} 

\includeonly{part/container}
\end{verbatim}

Once we are satisfied, we either remove the include only line, or we
comment it out.
